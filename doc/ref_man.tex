\documentclass[11pt]{article}

\usepackage{a4wide}
\usepackage{html}

\begin{htmlonly}
\newcommand{\op}{\Math{@}}
\newenvironment{block}{}{}
\newcommand{\sub}[2]{#1\_{\Math{#2}}}
\newcommand{\primesub}[2]{#1\_\Math{#2'}}
\newcommand{\Cdots}{\textrm{\ldots}}
\end{htmlonly}

% Latex version of the above:
% The latexonly environment doesn't work for this purpose.
\latex{\newcommand{\op}{$\oplus$}
\newenvironment{block}{\begin{tabular}[t]{@{}l@{}}}{\end{tabular}}
\newcommand{\sub}[2]{$\mbox{#1}_{#2}$}
\newcommand{\primesub}[2]{$\mbox{#1}'_{#2}$}
\newcommand{\Cdots}{$\cdots$}
}

% The rest is common to both.

\newcommand{\Hope}{{\scshape Hope}}

\input{op}

% latex2html doesn't know all of these
\newcommand{\Meta}[1]{\textit{\textmd{\textrm{#1}}}}
\newcommand{\Math}[1]{\textit{\textmd{\textrm{#1}}}}
\newcommand{\Samp}[1]{\textup{\textmd{\texttt{#1}}}}
\newcommand{\N}[1]{\mbox{#1}}

\newenvironment{commands}%
{\begin{description}\latex{\itemsep 0pt plus 1pt \parsep 0pt plus 1pt}}%
{\end{description}}

\newcommand{\cmditem}[1]{\item[\Samp{#1}]\latex{\hfill \\}}

\newcommand{\Section}[2]{\section{#2}\label{sec:#1}}
\newcommand{\SubSection}[2]{\subsection{#2}\label{sec:#1}}
\newcommand{\sectionref}[2]{\hyperref{#1}{#1 (see section }{)}{#2}}

\newcommand{\optional}[1]{\mbox{{\rm [}}{#1}\mbox{{\rm ]}}}

\newcommand{\expr}{\htmlref{\Meta{expression}}{sec:expressions}}
\newcommand{\pat}{\htmlref{\Meta{pattern}}{sec:patterns}}
\newcommand{\type}{\htmlref{\Meta{type}}{sec:types}}
\newcommand{\ident}{\htmlref{\Meta{identifier}}{sec:identifiers}}
\newcommand{\num}{\htmlref{\Meta{numeric-literal}}{numbers}}
\newcommand{\identX}[1]{\htmlref{\Meta{#1-identifier}}{sec:identifiers}}

\newcommand{\identM}{\identX{module}}
\newcommand{\identT}{\identX{type}}
\newcommand{\identTC}{\identX{type-constructor}}
\newcommand{\identC}{\identX{data-constant}}
\newcommand{\identCC}{\identX{data-constructor}}
\newcommand{\identV}{\identX{value}}

\newcommand{\Standard}
{\hyperref{\Samp{Standard}}{\Samp{Standard} (see appendix }{)}{sec:standard}}

\newenvironment{addition}{}{}

\begin{document}

\title{A Hope Interpreter --- Reference}
\author{Ross Paterson}
\maketitle

%\tableofcontents

This manual is not a tutorial on functional programming,
or on the language \Hope.
If you don't know about both, you might start with something like
Roger Bailey's tutorial \cite{tutorial}.

\Section{lexical}{Lexical structure}

The input is divided into a sequence of symbols of the following kinds:
\begin{itemize}
\item
a punctuation character: one of ``\Samp{( ) , ; [ ]}'',
\item
an identifier:
either
\begin{enumerate}
\item
a letter or underscore followed by zero or more letters, underscores or digits
(followed by zero or more single quotes), or
\item
a sequence of graphic characters that are neither white-space, letters,
underscores, digits, punctuation nor one of ``\Samp{\N{!} " '}'',
\end{enumerate}
\item
\label{numbers}a numeric literal:
a sequence of digits
(on some systems, optionally with an embedded decimal point
and an optionally signed exponent),
\item
a character literal:
either a single character (but not a newline),
or a character escape sequence (as in the C language),
enclosed in single quotes, or
\item
a string literal:
a sequence of zero or more characters
(but not newlines of double quotes) or character escape sequences,
enclosed in double quotes.
\end{itemize}

Symbols may be separated by white-space and comments.
A comment is introduced by `\Samp{!}' and continues to the end of the line.

The following identifiers are reserved by the language:
\begin{verbatim}
      ++ --- : <= == => |
      abstype data dec display else edit exit if in
      infix infixr lambda let letrec private save then
      type typevar uses where whererec write
\end{verbatim}
The following identifers are also reserved,
for compatability with other implementations:
\begin{verbatim}
      end module nonop pubconst pubfun pubtype
\end{verbatim}
Also, the following synonyms are available:
\verb|\| for \Samp{lambda},
\Samp{use} for \Samp{uses} and
\Samp{infixrl} for \Samp{infixr}.

\Section{identifiers}{Identifiers}

Identifiers may refer to
\begin{itemize}
\item
modules,

\item
types,

\item
type constructors,

\item
data constants,

\item
data constructors, or

\item
values.
\end{itemize}
The same identifier may refer to more than one of these kinds,
but in any scope it cannot refer to more than one of the last three classes.

Identifiers may be defined as \label{operators}\emph{infix operators},
by \Samp{infix} and \Samp{infixr} \sectionref{definitions}{sec:definitions}.
This affects only the way in which they are parsed and printed:
if an identifier \op{} is declared as an infix operator,
constructs of the form \Samp{\op(\Math{a}, \Math{b})}
are written \Math{a} \op{} \Math{b}.
To refer to \op{} on its own, use \Samp{(\op)}.
A special case: in any subsequent declaration of \op{} as a data constructor,
\Samp{\op(\sub{\type}{1}{} \# \sub{\type}{2})}
is written \sub{\type}{1}{} \op{} \sub{\type}{2}.

A number of identifiers are predefined in the module \Standard.

\Section{composite}{Composite structures}

In the following description,
constant width text (e.g. \Samp{dec} or \Samp{++})
denotes literal program text,
while italic text (e.g. \identT{} or \pat)
denotes syntactic classes.
Non-terminals are defined in sections of the same name.
The brackets \optional{ and } are used around optional text.
Alternative forms are used if identifiers are declared as
\sectionref{operators}{operators}.

\SubSection{modules}{Modules}

A \Hope{} module \Math{name} is a text file \Math{name}\Samp{.hop}
consisting of definitions.
The definitions may occur in any order,
but identifiers must be appropriately declared before use.

There is a special \Hope{} module called \Standard,
which is implicitly imported by every other \Hope{} module and session.

An interactive session consists of \sectionref{definitions}{sec:definitions}
together with other \sectionref{commands}{sec:commands}.

\SubSection{definitions}{Definitions}

\begin{commands}

\cmditem{uses \identM, \ldots, \identM;}
all the definitions visible in the named modules are to be visible in
the current module or session.
The modules are searched for first in the current directory,
and then in a library directory.
Used modules may use other modules, and so on, provided there are no cycles.
Definitions visible in the using module (or session)
will not be visible in a used module,
unless the definitions reside in a third module used by both.

\cmditem{private;}
all subsequent definitions in this module will be hidden from
other modules using it.
In particular, the visible definitions in a module used after this directive
will not be visible to any module using this one,
whereas they would be if the module was used before the directive.

\cmditem{infix \ident, \ldots, \ident{} \N{:} \num;}
declare left associative infix \htmlref{operators}{operators}
with the stated precedence,
from \Samp{\MINPREC} (weakest) to \Samp{\MAXPREC} (strongest).

\cmditem{infixr \ident, \ldots, \ident{} \N{:} \num;}
like \Samp{infix}, except that the \htmlref{operators}{operators}
are right associative.

\cmditem{abstype \ident{} \optional{(\sub{\ident}{1}, \ldots, \sub{\ident}{n})};}
an `abstract' type definition,
declaring the \ident{} as a type or type constructor identifier,
which may be used in subsequent definitions.
The \ident{} may be defined later by a \Samp{data} or \Samp{type} definition.
If there is only one argument, the parentheses may be omitted.

\cmditem{data \ident{} \begin{block}
	\optional{(\sub{\ident}{1}, \ldots, \sub{\ident}{n})} == \\
	\ \ \ \ \ \ \primesub{\ident}{1} \optional{\sub{\type}{1}} ++ \ldots{} ++
		\primesub{\ident}{k} \optional{\sub{\type}{k}};
\end{block}}
define the \ident{} as a type or type constructor,
with the \primesub{\ident}{k} as its data constants and constructors.
If there is only one simple argument, the parentheses may be omitted.

The definition may be recursive.
Any uses of \ident{} in \sub{\type}{i} should have the same parameters as the
left-hand side.

\begin{addition}
Any constructor identifier may have been previously declared
in a \Samp{dec} declaration,
in which case the new type of the identifier must be an instance
of that given in the prior declaration.
\end{addition}

\cmditem{type \ident{} \optional{(\sub{\ident}{1}, \ldots, \sub{\ident}{n})} == \type;}
define the \ident{} as an abbreviation for the \type,
which may refer to the argument type identifiers.
If there is only one simple argument, the parentheses may be omitted.

%The definition must not be recursive,
%but indirect recursion (using \Samp{abstype} definitions) is permitted,
%provided a type constructor defined with \Samp{data} intervenes.

\begin{addition}
The definition may be recursive.
Any uses of \ident{} in \type{} should have the same parameters as the
left-hand side.
See \hyperref{Regular types}{section }{}{sec:regular} for more details.
\end{addition}

\cmditem{typevar \ident, \ldots, \ident;}
declare new type variables,
for use in \Samp{dec} declarations.

\begin{addition}
If \ident{} is declared, then \ident\Samp{'}, \ident\Samp{''} and so on
may also be used as type variables
(single quotes in \ident{} itself are ignored).
\end{addition}

\cmditem{dec \ident, \ldots, \ident{} \N{:} \type;}
declare the identifiers as value identifiers of the given type,
in which type variables declared by \Samp{typevar} definitions
stand for any type.

\begin{addition}
If only one identifier is being declared, the keyword \Samp{dec} is optional.
\end{addition}

\cmditem{--- \identV{} \sub{\pat}{1} \Cdots{} \sub{\pat}{n} <= \expr;}
define the value of an identifier.
If \Math{n} is zero, only one such definition is allowed for each identifier.
Otherwise, the identifier may be defined by one or more such definitions,
each with the same number of arguments.
See \hyperref{Semantics of pattern matching}{section }{ for the semantics}%
{sec:pattern-matching}.

\begin{addition}
The keyword \Samp{---} is optional.
\end{addition}
\end{commands}

\SubSection{types}{Types}

\begin{commands}
\cmditem{\identT}
the type referred to.

\cmditem{\identTC(\type, \ldots, \type)}
a constructed type or type abbreviation.
If there is only one type argument, the parentheses may be omitted.

\cmditem{(\type)}
same as \type.
\end{commands}

\SubSection{patterns}{Patterns}

\begin{commands}
\cmditem{\ident}
(but not a \identC)
a variable, which matches any value.
No variable may occur twice in the same pattern.
Each free occurrence of the \ident{} in the expression corresponding
to the pattern is bound by this pattern.
That is, the identifier is a \identV{} in the expression,
referring to the value matched.

\cmditem{\identC}
matches the named data constant.

\cmditem{\identCC{} \pat}
matches a value formed by applying the data constructor
to a value matching \pat.

\cmditem{\num}
\Samp{1}, \Samp{2}, \ldots{} are abbreviations for
\Samp{succ(0)}, \Samp{succ(succ(0))} etc.

\cmditem{\pat{} + \num}
The form \pat\Samp{+}\Math{k} is an abbreviation for
\Samp{succ} applied \Math{k} times to \pat.

\cmditem{'\Math{c}'}
matches the character constant.

\cmditem{"\Meta{string}"}
abbreviation for a list of characters.

\cmditem{[\sub{\pat}{1}, \ldots, \sub{\pat}{n}]}
equivalent to
`\Samp{\sub{\pat}{1} \N{::} \ldots{} \mbox{::} \sub{\pat}{n} \N{::} nil}'.

\cmditem{[]}
equivalent to `\Samp{nil}'.

\cmditem{\sub{\pat}{1}, \sub{\pat}{2}}
matches a pair of values matching \sub{\pat}{1} and \sub{\pat}{2} respectively.

\cmditem{(\pat)}
same as \pat.
\end{commands}

\SubSection{expressions}{Expressions}

\begin{commands}
\cmditem{\identV}
the value bound to the identifier.

\cmditem{\identC}
a data constant.

\cmditem{\identCC}
a data constructor.

\cmditem{\num}
\Samp{1}, \Samp{2}, \ldots{} are abbreviations for
\Samp{succ(0)}, \Samp{succ(succ(0))} etc.

\cmditem{'\Math{c}'}
character constant.

\cmditem{"\Meta{string}"}
abbreviation for a list of characters.

\cmditem{[\sub{\expr}{1}, \ldots, \sub{\expr}{n}]}
equivalent to
`\Samp{\sub{\expr}{1} \N{::} \ldots{} \N{::} \sub{\expr}{n} \N{::} nil}'.

\cmditem{[]}
equivalent to `\Samp{nil}'.

\cmditem{\sub{\expr}{1}, \sub{\expr}{2}}
a pair formed from the values of \sub{\expr}{1} and \sub{\expr}{2}.

\cmditem{(\expr)}
same as \expr.

\cmditem{\sub{\expr}{1} \sub{\expr}{2}}
the result of applying the function or constructor value of \sub{\expr}{1}
to the value of \sub{\expr}{2}.

%\cmditem{lambda \begin{tabular}[t]{@{}l@{}}
%	$\pat_{1 1}$ \Cdots{} $\pat_{1 n}$ => \sub{\expr}{1} | \Cdots{} | \\
%	$\pat_{k 1}$ \Cdots{} $\pat_{k n}$ => \sub{\expr}{k}
%\end{tabular}}
\cmditem{lambda \sub{\pat}{1} => \sub{\expr}{1} | \Cdots{} | \sub{\pat}{k} => \sub{\expr}{k}}
an anonymous function.
See \hyperref{Semantics of pattern matching}{section }{ for the semantics}%
{sec:pattern-matching}.

\cmditem{if \sub{\expr}{1} then \sub{\expr}{2} else \sub{\expr}{3}}
a conditional expression,
equal to \sub{\expr}{2} if \sub{\expr}{1} is \Samp{true},
or \sub{\expr}{3} if it is \Samp{false}.

\cmditem{let \pat{} == \sub{\expr}{1} in \sub{\expr}{2}}
the same as \sub{\expr}{2}, with the variables in \sub{\expr}{2} replaced by the values
assigned to them in matching \pat{} to \sub{\expr}{1}.
It is equivalent to
\begin{quotation}
\Samp{(lambda \pat{} => \sub{\expr}{2}) \sub{\expr}{1}}
\end{quotation}

\cmditem{letrec \pat{} == \sub{\expr}{1} in \sub{\expr}{2}}
like \Samp{let},
except that \pat{} must be irrefutable,
and its variables may also appear in \sub{\expr}{1}.
It is a more efficient version of
\begin{quotation}
\Samp{(lambda \pat{} => \sub{\expr}{2})(fix(lambda \pat{} => \sub{\expr}{1}))}
\end{quotation}
for a function \Samp{fix} defined as
\begin{verbatim}
      dec fix: (alpha -> alpha) -> alpha;
      --- fix f <= f(fix f);
\end{verbatim}

\cmditem{\sub{\expr}{1} where \pat{} == \sub{\expr}{2}}
same as `\Samp{let \pat{} == \sub{\expr}{2} in \sub{\expr}{1}}'.

\cmditem{\sub{\expr}{1} whererec \pat{} == \sub{\expr}{2}}
same as `\Samp{letrec \pat{} == \sub{\expr}{2} in \sub{\expr}{1}}'.
\end{commands}

Ambiguities in patterns and expressions are resolved by
the following binding precedences, from weakest to strongest:
\begin{itemize}
\item comma (right associative)
\item \Samp{lambda}
\item \Samp{let} and \Samp{letrec}
\item \Samp{where} and \Samp{whererec}
\item \Samp{if}
\item infix \htmlref{operators}{operators} of precedence \MINPREC to \MAXPREC
\item function application (left associative)
\end{itemize}
\begin{addition}
For any infix \htmlref{operators}{operators} \op{} and expression \Math{e},
the following abbreviations, called \emph{operator sections}, are permitted:
\begin{description}
\item[\Samp{(\Math{e} \op)}]
is short for \Samp{lambda x => \Math{e} \op{} x}.

\item[\Samp{(\op{} \Math{e})}]
is short for \Samp{lambda x => x \op{} \Math{e}}.
\end{description}
\end{addition}

\SubSection{commands}{Interactive commands}

In a \Hope{} session, any \sectionref{definitions}{sec:definitions}
may be entered (although \Samp{private} will be ignored),
as well as the following commands:
\begin{commands}
\cmditem{\expr;}
display the value and most general type of \expr.
Lazy evaluation is used,
but nothing is displayed until the value is fully evaluated.

The \expr{} may involve the special variable \Samp{input},
which denotes the list of characters typed at the terminal.

\cmditem{write \expr~\optional{to "\Math{file}"};}
output the value of \expr{} (which must be a list of values) more directly:
if the value is a list of characters, the characters are printed;
otherwise each value is printed on a line by itself.
Each element of the list is printed as soon as its value is computed
(by lazy evaluation).
If the \Samp{to} clause is present, the output is written to the file;
otherwise it is printed on the screen.
It is safe to write to a file that is read by the expression---no
interference will occur.

The \expr{} may involve the special variable \Samp{input},
which denotes the list of characters typed at the terminal.

\cmditem{display;}
display the \htmlref{definitions}{sec:definitions} of the current session.

\cmditem{save \identM;}
save the \htmlref{definitions}{sec:definitions} of the current session
in a new \htmlref{module}{sec:module} of the given name.
The \htmlref{definitions}{sec:definitions} are replaced in the current session
by a reference to the new module.

\cmditem{edit \optional{\identM};}
(Unix only)
invoke the default editor on
the named \htmlref{module}{sec:modules}, if given,
or otherwise on a file containing the current definitions,
and then re-enter the interpreter with the revised definitions.

\cmditem{exit;}
exit from the interpreter.
\end{commands}

\Section{pattern-matching}{Semantics of pattern matching}

The following is a partial specification; nothing more is guaranteed.
Informally, when patterns overlap,
more specific patterns are preferred to more specific ones.
Apart from that, the choice is unspecified.
More formally:

Matching a value with a pattern involves evaluating the value sufficiently
to compare its data constants and constructors with those in the pattern,
in some consistent, top-to-bottom (but otherwise unspecified) order.
This match may result in
\begin{description}
\item[success:]
the value can be seen to be an instance of the pattern,

\item[failure:]
the value cannot be an instance of the pattern,
because it has a different data constant or constructor in some position, or

\item[non-termination:]
the value cannot be sufficiently evaluated to decide between the above,
in the particular order of checking used.
\end{description}
If the match succeeds, it supplies values for the variables of the pattern.

A pattern (or sub-pattern) consisting only of variables and pairs
always matches a value of the appropriate type,
even if computation of the value would not terminate.
Such patterns are called \emph{irrefutable}.

Now suppose a function has been defined by a series of definitions
\begin{quotation}
	\Samp{--- \Math{f} \sub{\Math{p}}{i1} \Cdots{} \sub{\Math{p}}{in} <= \sub{\Math{e}}{i};}
\end{quotation}
for \Math{i} = 1, \ldots, \Math{k},
or is the expression
\begin{quotation}
	\Samp{lambda
	\sub{\Math{p}}{11} \Cdots{} \sub{\Math{p}}{1n} => \sub{\Math{e}}{1}
	| \Cdots{} |
	\sub{\Math{p}}{k1} \Cdots{} \sub{\Math{p}}{kn} => \sub{\Math{e}}{k}
	}
\end{quotation}
(At present, anonymous functions are required to be unary.)
An application of this function to \Math{n} argument values
will have zero or more possible values, as follows:
\begin{itemize}
\item
If some patterns \sub{\Math{p}}{ij}
successfully match the corresponding arguments for some \Math{i},
and all more specific patterns (if any) fail,
then a possible value is the value of \sub{\Math{e}}{i},
with the variables of \sub{\Math{p}}{ij} taking the values they matched.

\item
If the matching of some pattern does not terminate,
under some order of checking,
then non-termination is a possible value.
\end{itemize}
If there are no possible values,
the value of the application is a run-time error.
Otherwise, one of the possible values is chosen,
in a deterministic (but otherwise unspecified) manner.

\bibliographystyle{html-plain}

\bibliography{hope}

\nocite{hope}
\nocite{tutorial}
\nocite{field&harrison}
\nocite{bailey}

\appendix

\clearpage

\Section{standard}{Appendix: The Standard module}

\input{Standard}

\clearpage

\Section{deviations}{Deviations from other versions of the language}

The following features are special to this version of \Hope:
irrefutable patterns,
operator sections,
functions of more than one argument,
recursive \Samp{type} definitions,
\Samp{input}, \Samp{read}, \Samp{write},
\Samp{private}, \Samp{abstype},
\Samp{letrec} and \Samp{whererec}.
Other versions may not support full lazy evaluation,
and may have several features not provided here.

This version also supports curried type and data constructors.

The rest of this appendix describes some experimental features of this
version of the language.

\SubSection{regular}{Regular types}

The type system is generalized to permit \emph{regular} types,
which are possibly infinite types that are unrollings of finite trees.
For example, this makes possible a definition of Curry's Y combinator:
\begin{verbatim}
      dec Ycurry : (alpha -> alpha) -> alpha;
      --- Ycurry f <= Z Z where Z == lambda z => f(z z);
\end{verbatim}
This fails in the usual type system because \Samp{Z} must have a function
type with argument type the same as the whole type.

Type synonyms are now allowed to be recursive,
so they can be used to describe infinite types,
like the type of infinite sequences:
\begin{verbatim}
      type seq alpha == alpha # seq alpha;
\end{verbatim}
Recursive uses of the type constructor being defined must have exactly
the same arguments as the left-hand side.
However, these arguments may be permuted, as in
\begin{verbatim}
      type alternate alpha beta == alpha # alternate beta alpha;
\end{verbatim}
The syntax of types is also extended to express regular types.
For example, the above two types could be defined as
\begin{verbatim}
      type seq alpha == mu s => alpha # s;
      type alternate alpha beta == mu a => alpha # (beta # a);
\end{verbatim}
This notation is also used by the system to print regular types.

\SubSection{functors}{Functors}

Each \Samp{data} or \Samp{type} definition introduces a new `map' function
(or functor, for category theorists)
with the same name and arity as the type constructor.
For example, the type definition
\begin{verbatim}
      data tree alpha == Empty ++ Node (tree alpha) alpha (tree alpha);
\end{verbatim}
also defines a function
\begin{verbatim}
      tree : (alpha -> beta) -> tree alpha -> tree beta
\end{verbatim}
that maps a function over trees.
This automatic definition may be explicitly overridden.

It's a little more complicated:
if the type argument is used negatively, as in
\begin{verbatim}
      type cond alpha == alpha -> bool;
\end{verbatim}
the function will have a flipped type:
\begin{verbatim}
      cond : (alpha -> beta) -> tree beta -> tree alpha
\end{verbatim}
If the argument is used both positively and negatively, as in
\begin{verbatim}
      type endo alpha == alpha -> alpha;
\end{verbatim}
the function will have a type like
\begin{verbatim}
      endo : (alpha -> alpha) -> tree alpha -> tree alpha
\end{verbatim}
Similarly, an \Samp{abstype} definition declares this corresponding function.
In order to determine the type,
each argument is assumed to be used both positively and negatively
unless the argument variable is replaced with one of the special keywords
\Samp{pos}, \Samp{neg} or \Samp{none} (indicating that the argument is unused).
See the previous appendix for some examples.

\end{document}
